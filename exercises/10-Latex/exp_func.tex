\documentclass{article}
\usepackage{amsmath}
\usepackage{graphics}
\usepackage{listings}
\title{Implementation of exponential function in C}
\author{Martin Aagaard}
\date{}
\begin{document}
\maketitle 
\begin{abstract}
In this short report an implementation of the exponential function
will be investigated.  
\end{abstract}

\section{The exponential function}
The exponential function is widely known and used and can be defined through the
property that $\frac{d y(x)}{dx} = y$ if $y(x)=e^x$. The exponential function
appears many places e.g. in physics which necessitates an implementation in
computer languages. One way to do this is through the taylor series of the
exponential function, 

\begin{align}
e^x=\sum^{\infty}_{n=0} \frac{x^n}{n!} \label{Taylor_exp} \; . 
\end{align}

\section{Implementation}
From the taylor series it is seen that one can take factor out factors of x and 
n leading to an expression on the following form:
\begin{align}
1+x\cdot(1+\frac{x}{2}\cdot(1+\frac{x}{3}\cdot(...))) \; .
\end{align}
This odd form of writing the series minimizes the number of time a program would
need to multiply and through that minimizing the operation time. 
In C a final implementation could look like:
\begin{lstlisting}[language=C]
double ex(double x){
if(x<0)return 1/ex(-x);
if(x>1./8)return pow(ex(x/2),2);
return 1+x*(1+x/2*(1+x/3*(1+x/4*(1+x/5*(1+x/6*(1+x/7*(1+
x/8*(1+x/9*(1+x/10)))))))));
} . 
\end{lstlisting}
Here the if statements make sure that the value of x is significantly small. 
This allows us to include fewer terms of the taylor series.

\section{Result}
The implementation is now compared to the exp-function from math.h. The result
is seen in figure~(\ref{fig:exp})
	\begin{figure}
\input{exp_plot}
\caption{Implementated exponential function (Implexp) compared with the math.h 
implemantation(exp) via gnuplot "latex" terminal.}
\label{fig:exp}
	\end{figure} 

\end{document}

